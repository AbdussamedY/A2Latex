\documentclass[12pt]{article}

\usepackage{A2SUP}

\informations{
                PASSLAS = {PASS},
                     UE = {9},
                Support = {tuto},
        NumeroDuSupport = {1},
      SujetOuCorrection = {correction},
          Coordinateurs = {},
             Redacteurs = {},
             Relecteurs = {},
    ConvertisseursLatex = {},
        TuteursEnSeance = {},
               Testeurs = {},
     RecommandationsSpe = {},
             TitreExtra = {},
             TitreCours = {},
      NombreDeQuestions = {},
   FacRedactionOuCollab = {},
            ProfDuCours = {},
     RemerciementsBonus = {},
             TiersTemps = {},
    CorrectionItemIsole = {}
    }

\begin{document}


\textcolor{red}{\lipsum[1]}

\inseparable % tout ce qui suit cette commande sera inséparable de la question du dessous, pas de saut de page entre les deux donc (optionnel)


\textcolor{orange}{\lipsum[2] introduisant la question (donc inséparable de la qst)}

\question{1}
{Parmi les propositions suivantes, laquelle est (ou lesquels sont) exacte(s) ?}
{Je suis l'item A}
{Je suis l'item B}
{Je suis l'item C}
{Je suis l'item D}
{Je suis l'item E}

\vrai{ABE}

\correction{
\correctionitem{Item A}

\correctionitem{Item B}

}

\end{document}