\documentclass[12pt]{article}

% lien vers documentation : https://drive.google.com/drive/folders/1nm1PBXNYbQI7Wz4dUtG4qTWBp9g9JSPI?usp=sharing
\usepackage{A2SUP}

\begin{informations}
    UE = 6,
    Support = extra,
    NumeroDuSupport = 1,
    SujetOuCorrection = ,
    Orientation = ,
    Pagination = ,
    TiersTemps = non,
    RecoSpecifiques =,
    TitreExtra = ,
    SousTitre = ,
    Team = SPectaculaire,
    FacRedactionOuCollab = collab,
    RemerciementsBonus =,
    Session = ,
    RelectureProfs = ,
    Redacteurs = {
        % NOM Prénom (filière, poste si CA),
    },
    Relecteurs = {
        % NOM Prénom (filière, poste si CA),
    },
    ConvertisseursLatex = {
        % NOM Prénom (filière, poste si CA),
    },
    Testeurs = {
        % NOM Prénom (filière, poste si CA),
    },
    TuteursEnSeance = {
        % NOM Prénom (filière, poste si CA),
    },
    VP = {
        Lùca BLONDEL-JORAND (DFGSM2) et 
        Souad BENABDI (DFGSM2)
        % Eugénie Gauthier-Lescop (DFGSM2) et
        % Suzon Vivi (DFGSM2)
    },
    RM = {
        % NOM Prénom (filière, poste),
    },
\end{informations}










\begin{document}

\recto
\grille


\begin{motRM}[
        titre=Mot de votre RM,
        %%% OPTIONS
        % color=...,
        % afficher=toujours OU sujet OU correction (par défaut : correction)
    ]
    Ceci est une box pour y mettre un contenu important pour les étudiants 
\end{motRM}

\newpage


\section*{Exercice 1}

\inseparable % si je veux que l'énoncé reste toujours solidaire de la question 1, de façon à ce qu'un saut de page ne s'intercale pas entre les deux.

Ceci est un énoncé introductif de mon exercice 1. Ci-dessous, une équation :

$$
\int_{\mathbb R} e^{-x^2}\, dx = \sqrt{\pi}
$$

Et maintenant, voici une image :

\includegraphics[width=0.2\linewidth]{logo.png}

\question{1}
{
    Enoncé de la question 1, avec une image.
    
    \begin{center}
        \includegraphics[width=1cm]{logo.png}
    \end{center}
}
{Item A}
{Item B}
{Item C}
{Item D}
{Item E}
{ABCDE}

\begin{correction}
    \correctionitem{Item A}
    L'item A est \uline{vrai}. Si l'on pose $I=\int_{\mathbb R} e^{-x^2}\, dx$, alors :

    $$
    \begin{aligned}
        I^2 
            &= \int_{\mathbb R} e^{-x^2}\, dx \times \int_{\mathbb R} e^{-x^2}\, dx\\
            &= \int_{\mathbb R} e^{-x^2}\, dx \times \int_{\mathbb R} e^{-y^2}\, dy\\
            &= \int_{\mathbb R}\int_{\mathbb R} e^{-x^2}e^{-y^2}\, dxdy\\
            &= \iint_{\mathbb R^2} e^{-x^2-y^2}\, dxdy\\
            &= \iint_{\mathbb R^2} e^{-\left(x^2+y^2\right)}\, dxdy\\
            &= \int_{0}^{2\pi}\int_{0}^{+\infty} r\times e^{-r^2}\, dr d\theta
                \quad
                    \left(\text{car 
                        $\begin{cases}
                            x=r\cos(\theta)\\
                            y=r\sin(\theta)
                        \end{cases}$}
                    \right)\\
            &= \int_{0}^{2\pi}d\theta\int_{0}^{+\infty} r\times e^{-r^2}\, dr\\
            &= [\theta]_{0}^{2\pi}\left[-\frac{e^{-r^2}}{2}\right]_{0}^{+\infty}\\
            &= \left(2\pi - 0\right) \times \left\{-\lim_{x\to +\infty} \left(\frac{e^{-r^2}}{2}\right) + \frac{1}{2}\right\}\\
            &= \pi\\
        \iff I &= \sqrt{I^2}\\
        &= \sqrt{\pi}
    \end{aligned}
    $$

    \correctionitem{Item B}
    Youhou
    
    \correctionitem{Item C}
    Youhou
    
    \correctionitem{Item D}
    Youhou
    
    \correctionitem{Item E}
    Youhou
\end{correction}




















\question{2}
{
    Enoncé
}
{Item A}
{Item B}
{
    Item C avec une image :

    \begin{center}
        \includegraphics[width=0.2\linewidth]{logo.png}
    \end{center}
}
{Item D}
{Item E}
{CE}


\begin{correction}
    \correctionitem{Item A}
    Youhou
    
    \correctionitem{Item B}
    Youhou
    
    \correctionitem*{Item C} % <--- Ne pas oublier * pour retirer le rappel d'item qui ferait ici une erreur puisqu'il y a une image intégrée dans l'item !!!
    Youhou
    
    \correctionitem{Items D et E}
    Youhou
\end{correction}

































\inseparable

\begin{center}
    \begin{tblr}{
        colspec = { c c c c c c c },
        hlines = {1pt, black},
        vlines = {1pt, black},
    }
        t (heure)
            & 0
            & 10
            & 20
            & 30
            & 40
            & 50\\ 
        $z(t)\, (\mu g/L)$
            & 0
            & 35
            & 32
            & 26
            & 22
            & 18\\ 
    \end{tblr}
\end{center}

\question{4}
{
    Enoncé
}
{Item A}
{Item B}
{Item C}
{Item D}
{Item E}
{CE}

\begin{correction}
    \correctionitem{Item A, B, C, D et E}
    Youhou
\end{correction}



\verso

\end{document}